\documentclass{report}

% ! BIB TS-program = biber
\usepackage[utf8]{inputenc}
\usepackage[T1]{fontenc}
\usepackage{geometry}
\geometry{a4paper}

\usepackage{listings}
\lstset{frame=tb,	
  language=Java,
  aboveskip=3mm,
  belowskip=3mm,
  showstringspaces=false,
  columns=flexible,
  basicstyle={\small\ttfamily},
  numbers=none,
  breaklines=true,
  breakatwhitespace=true,
  tabsize=3
}

\usepackage[backend=biber]{biblatex}
\addbibresource{Project Documentation.bib}

\title{mars - a CLI tool}
\author{Mario Fernández}


\begin{document}
\maketitle

\tableofcontents

\chapter{Introduction}
\section{About the author}
The entirety of the project, including the design, planning and implementation is maintained by the author, @mariolfh on Github \cite{githubinc.MariolfhOverview2025}. As a computer science graduate, getting into contact with all sort of concepts and ideas was something marvelous. So, in the hopes of having several important and key tools at hand, mixed with a love for terminals, made the original thought that would turn into this project. 

\section{Objectives}
This project follows two main objectives:
\begin{enumerate}
\item{Learn and practice new technologies.}
\item{Showcase my abilities with certain prefered technologies.}
\item{Share some of my personal tools / project ideas with the world.}
\end{enumerate}

\section{Technologies}
This project wanted to be low maintenance and keep it simple, something great for the command line to do. For this task a low level language was needed, in this case, Rust \cite{RustProgrammingLanguage} was the best option. Both a personal like and something new, it proved worthy of the task. For the documentation you're reading, \LaTeX\ \cite{LaTeXDocumentPreparation} came along, as it was a technology i wanted to learn since college and that  has proved worthy of taking. Finally but not least, this is all being saved by a version control system, which in this case is Git \cite{Git}.

\chapter{Requirements}
\section{Functional requirements}
(Table with FR)
\section{Non functional requirements}
(Table with NFR)

\chapter{Functionalities}
\section{Commands}
This chapter goes deep into the usage of \emph{mars} \cite{fernandezMariolfhMars2025}.  From code snippets and usage to examples, you'll get it all. 

\subsection{Welcome}
This function prints a welcoming message to the user. Firstly thought as a meme or joke, it stayed as one of the first functions to be written for the app. It also served as a test run to check if the app worked at all.
\subsubsection{Usage}
\begin{lstlisting}
// Basic command call
$ mars welcome
Welcome to Mars!
\end{lstlisting}

\subsection{Mars}
This function prints the main inspirational quote for the emph{mars} project, said by Buzz Aldrid, an American pilot and astronaut. It's all there, at reach, waiting for us to get there, no matter how much time it takes, you just gotta make the effort to get there. It's very inspiriing to myself.
\subsubsection{Usage}
\begin{lstlisting}
// Basic command call
$ mars mars
``Mars is there, waiting to be reached.'' - Buzz Aldrin, American pilot and astronaut, 2009
\end{lstlisting}

\subsection{Uppercase}
This function requires a string as a parameter. This string is taken by the tool and transforms it into a fully uppercase string. This is very useful to myself, I've needed several times to work with titles in uppercase that were so long to type out again in lowercase, so this is here just for that.
\begin{lstlisting}
// Basic command call
$ mars uppercase <string>
<STRING>
\end{lstlisting}

\subsection{Lowercase}
This function requires a string as a parameter. This string is taken by the tool and is returned with all characters in uppercase. This saved lots of work during college, in which I needed to present author names in lowercase for my papers. Back then it was written in Python and just sitting around in a scripts folder, but now it's here as a part of something much bigger.
\begin{lstlisting}
// Basic command call
$ mars lowercase <STRING>
<string>
\end{lstlisting}

\chapter{Final Considerations}
\section{Closing thoughts}
I wanted to take the time to express myself down here, and so, i want to thank several people who i won't mention for believing in me and for boosting to come back into my field and heart. Also, I want you, the reader, to take this as a sign that something brighet and big is out there for you, even if you think it hasn't come or it isn't your turn yet. Guess what? You make your turn! I did the same thing with this project, my first project, on my own. Doing it, scared and excited, but doing it. Go after the things you like and want to know. The only thing stopping you is yourself.

\printbibliography[title=Citations]


\end{document}